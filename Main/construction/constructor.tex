\documentclass[a4paper,12pt,dutch]{article}
\usepackage{glossaries}
\usepackage[T1]{fontenc}
\usepackage{babel}
\usepackage{graphicx}
\usepackage[table,xcdraw]{xcolor}
\usepackage{hyperref}
\usepackage{blindtext}
\usepackage{geometry}
\usepackage{parskip}
\usepackage{mathtools}
\usepackage{siunitx}
\usepackage{listings}
\usepackage{csquotes}
\usepackage{caption}
\usepackage{subcaption}
\usepackage{comment}
\usepackage{pdfpages}
\usepackage{float}
\usepackage{pict2e}
\usepackage{tabularx}
\usepackage{xcolor}
\usepackage{booktabs}
\usepackage{threeparttable}


% \DeclareRobustCommand{\slashcirc}{{\mathpalette\doslashcirc\relax}}

% \makeatletter
% \newcommand\doslashcirc[2]{%
%   \sbox\z@{$#1\m@th\circ$}%
%   \setlength\unitlength{\wd\z@}
%   \begin{picture}(1,1)
%   \roundcap
%   \put(0,0){\box\z@}
%   \put(0,0){\line(1,1){1}}
%   \end{picture}%
% }
% \makeatother


% %% Some packages you will need
% \usepackage{pgfplots}
% \usepackage{pgfplotstable}
% \usepackage{booktabs}
% \usepackage{array}
% \usepackage{colortbl}


\definecolor{arduinoorange}{HTML}{FFA500}
\definecolor{arduinogray}{HTML}{808080}
\definecolor{arduinoblue}{HTML}{007ACC}
\definecolor{arduinogreen}{HTML}{469B00}

\lstset{
  language=C++,
  basicstyle=\ttfamily\footnotesize,
  keywordstyle=\color{arduinoorange},
  stringstyle=\color{arduinogreen},
  commentstyle=\color{arduinogray},
  moredelim=[s][\color{arduinoblue}]{\#}{\ },
  morekeywords={digitalRead,digitalWrite,pinMode,analogRead,analogWrite,Serial,begin,HIGH,LOW},
  frame=tb,
  tabsize=4,
  showstringspaces=false,
  breaklines=true,
  numbers=left,
  numberstyle=\tiny\color{arduinogray},
  numbersep=5pt,
  extendedchars=true,
  literate={á}{{\'a}}1 {ã}{{\~a}}1 {é}{{\'e}}1,
}

\lstdefinestyle{Arduino}
{
  language=C++,
  basicstyle=\ttfamily\footnotesize,
  keywordstyle=\color{arduinoorange},
  stringstyle=\color{arduinogreen},
  commentstyle=\color{arduinogray},
  moredelim=[s][\color{arduinoblue}]{\#}{\ },
  morekeywords={digitalRead,digitalWrite,pinMode,analogRead,analogWrite,Serial,begin},
  frame=tb,
  tabsize=4,
  showstringspaces=false,
  breaklines=true,
  numbers=left,
  numberstyle=\tiny\color{arduinogray},
  numbersep=5pt,
  extendedchars=true,
  literate={á}{{\'a}}1 {ã}{{\~a}}1 {é}{{\'e}}1,
  backgroundcolor=\color{black!85},
  rulecolor=\color{arduinoorange},
  frame=single,
  frameround=tttt,
  framexleftmargin=6pt,
  framexrightmargin=6pt,
  framextopmargin=6pt,
  framexbottommargin=6pt,
  breaklines=true,
  postbreak=\raisebox{0ex}[0ex][0ex]{\ensuremath{\color{red}\hookrightarrow\space}},
}

\usepackage[
  style=apa,
  backend=biber,
  backref=true,
  backrefstyle=none,
  sortcites=true,
  sorting=nyt,
  doi=false,
  maxcitenames=3,
  language=american
]{biblatex}
\DeclareLanguageMapping{dutch}{dutch-apa}
\addbibresource{Main/construction/Sources.bib}

\DefineBibliographyStrings{dutch}{
    backrefpage = {blz.},
    backrefpages = {blz.},
}
\makeglossaries
\definecolor{Grey1}{HTML}{343434}
\graphicspath{{./Media/Figuren/}}
 \geometry{
 a4paper,
 total={170mm,257mm},
 left=20mm,
 top=20mm,
 }
\hypersetup{
    colorlinks=true,
    linkcolor=blue,
    filecolor=magenta,      
    urlcolor=cyan,
    pdftitle={Overleaf Example},
    pdfpagemode=FullScreen,
    }

\sisetup{
  range-phrase = --,   % en-dash i.p.v. "to"
  range-units  = single, % zet de eenheid één keer achteraan
  table-number-alignment = center,
  table-text-alignment  = center
}



\begin{document}
\title{
\includegraphics[width=3.5in]{Main/assets/logo.jpg} \\
\vspace*{1in}
\textbf{Modelleren}\\
\textit{Groepscasus 1}\\
Versie 1
}
\author{
\vspace*{0.3in} \\
  Geschreven door:\\
  Laurens van der Drift 21028605\\
  Chanamel Brigitha 25003534\\
  Daniël Grit 25152475\\
  Nieck Hoogteijling 25042742\\
  Abe Hoogveld 22134409\\
\vspace*{0.2in} \\
    The Hague University of Applied Sciences\\
    \textbf{Bewegingstechnologie}\\
    Den Haag, Nederland
   } 
\maketitle
\phantomsection
\section*{Versie Historie} \addcontentsline{toc}{section}{Versie Historie}

\begin{table}[h]
\begin{tabular}{|l|l|l|l|}
\hline
\rowcolor[HTML]{4472C4} 
{\color[HTML]{FFFFFF} \textbf{Version}} &
  {\color[HTML]{FFFFFF} \textbf{Date}} &
  {\color[HTML]{FFFFFF} \textbf{Changes}} &
  {\color[HTML]{FFFFFF} \textbf{Author}} \\ \hline
\rowcolor[HTML]{D9E1F2} 1.0 & \multicolumn{1}{c|}{\cellcolor[HTML]{D9E1F2}15-09-2025} & N.v.t. & N.v.t \\
\hline

% \rowcolor[HTML]{FFFFFF} 
% 2.0 &
%   \multicolumn{1}{c|}{\cellcolor[HTML]{FFFFFF}7-4-2023} &
%  Feedback van FeedbackFruits toegepast &
%   Infra   Vroom \\ \hline

% \rowcolor[HTML]{D9E1F2} 
% 3.0 &
%   \multicolumn{1}{c|}{\cellcolor[HTML]{D9E1F2}4-6-2023} &
%  Bijgewerkt voor Assessment 3 &
%   Infra   Vroom \\ \hline

\end{tabular}
\end{table}
\include{Main/construction/toc}
\phantomsection
\addcontentsline{toc}{section}{Lijst van Figuren}
\listoffigures


\section{Inleiding}
Men zegt dat je niet met je rug, maar met je benen moet tillen. Het argument gegeven voor dit advies is dat de belasting op de onderrug lager zou zijn bij een squat lift. Dit is een populaire opvatting, maar onderzoek toont aan dat het verschil in de belasting van de rug bij deze twee technieken niet zo duidelijk is. In dit onderzoek gaan wij 3 verschillende tiltaken uitvoeren. De stoop lift en 2 manieren van de squat lift. We gaan deze tiltaken analyseren om achter te komen wat het moment rond de onderrug (rond de L5) is. 	Door middel van dit onderzoek willen wij de volgende vraag beantwoorden.  Is het tillen ‘vanuit je benen’ inderdaad beter voor je lage rug dan tillen ‘vanuit je rug’?
\section{Werkwijze}
Voor het uitvoeren van dit onderzoek, hebben wij 4 proefpersonen gemeten. Elke proefpersoon moest een Velcro-pak aan dragen en daarop reflecterende markers erop plakken.  Deze zijn zodanig geplakt dat ze allemaal zichtbaar zijn vanuit het sagittale vlak. De punten waar markers werden geplakt is te vinden bij de onderstaande tabel.
\begin{figure}
    \centering
    \includegraphics[width=0.5\linewidth]{Main//Chapters//Deelvraag//assets/markers.png}
    \caption{Markers}
    \label{fig:Bodymarkers}
\end{figure}

De afstand tussen de knie en de enkel markers waren met een meetlint gemeten bij elke proefpersoon. Deze data werden later gebruikt om in Kinovea te kalibreren. 
\begin{figure}
    \centering
    \includegraphics[width=0.5\linewidth]{Main//Chapters//Deelvraag//assets/Kalibratie.png}
    \caption{Kalibratie}
    \label{fig:Kalibratie}
\end{figure}
De proefpersonen moesten dan 3 verschillende til technieken uitvoeren voor het tillen van een kratje.  De 3 til technieken zijn: Stooplift (met de rug tillen), Squatlift (met de benen tillen). Het squat tillen werd uitgevoerd op 2 manieren. Eerst squat tillen met het kratje voor de benen. Als tweede het squat tillen tussen de benen. Het uitvoeren van deze til bewegingen werden gefilmd dan geïmporteerd en analyseert in Kinovea.            
We hebben erop gelet dat elke proefpersoon de til bewegingen zo identiek mogelijk uitvoerden zodanig dat we goede en vergelijkbare resultaten kregen.
\begin{enumerate}
    \item De stoop lift: je tilt alsof je in een stoel gaat zitten. Je buigt je knieën en heupen   		 alsof je in een stoel gaat zitten, maar houd je rug recht.
    \item De squat voor knie: Je zorgt ervoor dat je tenen zo goed als tegen het kratje staat 			en maakt dan een squat beweging, je zorgt ervoor dat je een rechte rug houdt.
    \item De squat tussen knie: Je zorgt ervoor dat je voeten iets breder staan dan het 			      kratje. Je maakt weer een squat beweging en zorgt voor een rechte rug.
\end{enumerate}
 
Ook hebben we tijdens het filmen erop gelet dat de proefpersoon de beweging rustig uitvoert, zodat we tijdens het analyseren een goed punt/frame konden uitkiezen om te analyseren.
Dit geeft wel een statisch beeld weer en dit zou ervoor zorgen dat er een foute waarde van de belasting uit zou komen, maar als je net zoals wij de beweging rustig/langzaam uitvoert zullen deze verschillen minimaal zijn.

\begin{figure}
    \centering
    \includegraphics[width=0.5\linewidth]{Main/Chapters/Deelvraag/assets/Stooplift.png}
    \caption{Stooplift}
    \label{fig:Stooplift}
\end{figure}
\begin{figure}
    \centering
    \includegraphics[width=0.5\linewidth]{Main/Chapters/Deelvraag/assets/Squatlift.png}
    \caption{Squatlift voor}
    \label{fig:Squatlift_voor}
\end{figure}
\begin{figure}
    \centering
    \includegraphics[width=0.5\linewidth]{Main/Chapters/Deelvraag/assets/Squatlift_tussen.png}
    \caption{Squatlift tussen}
    \label{fig:Squatlift_tussen}
\end{figure}

In Kinovea hebben we de markers gedigitaliseerd en geexporteerd in Excel. Deze data werden dan in Spyder6 gebruikt, om de momenten bij de lage rug te berekenen. 


\include{Main/construction/glossary}
\include{Main/construction/sources}
\end{document}
