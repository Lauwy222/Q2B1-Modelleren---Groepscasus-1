\section{Wat zijn de benodigde afmetingen voor comfortabel gebruik van de fiets?}

Om een loopfiets comfortabel en ergonomisch verantwoord te maken, moeten de 
belangrijkste dimensies (zoals zadelhoogte, stuurhoogte en lengte van het frame) 
worden afgestemd op de doelgroep van kinderen tussen 9--12 jaar. Deze afmetingen 
worden vooral bepaald door de antropometrische gegevens van kinderen 
(\autoref{tab:anthropometry}) en de richtlijnen voor ergonomisch fietsen.

\subsection{Zadelhoogte}
De zadelhoogte is één van de meest bepalende factoren voor comfort en 
kniebelasting. Volgens \textcite{savelberg2003} moet de kniehoek tijdens de trapbeweging 
tussen de \SI{95}{\degree} en \SI{120}{\degree} liggen (zie ook \autoref{fig:fietsschema}). 
Dit betekent dat de zadelhoogte zodanig moet worden ingesteld dat kinderen met 
verschillende beenlengtes dit bereik behouden. Uit de data blijkt dat kinderen tussen 
9--12 jaar een binnenbeenlengte (\emph{inseam}) hebben van circa 
\SIrange{55}{75}{\centi\meter} \parencite{dined2023}. Een verstelbare zadelhoogte van 
ongeveer \SIrange{58}{72}{\centi\meter} wordt daarom aanbevolen.

\subsection{Stuurhoogte en stuurafstand}
Naast de zadelhoogte is ook de stuurhoogte en de afstand tot het stuur cruciaal voor 
een neutrale houding van rug en schouders. Volgens \textcite{conceicao2022} leidt een 
te laag stuur tot grotere armhoeken en verhoogde spierbelasting. Voor kinderen 
betekent dit dat het stuur enigszins hoger moet staan dan bij volwassenen, zodat 
een arm–romphoek van \SIrange{80}{100}{\degree} behouden blijft. De aanbevolen 
stuurhoogte ligt daarmee in de orde van \SIrange{80}{95}{\centi\meter}, afhankelijk van 
de lichaamslengte.

\subsection{Frame-afmetingen}
De lengte van het frame (afstand tussen zadel en stuur) bepaalt in hoeverre een kind 
comfortabel kan reiken zonder overbelasting. Richtlijnen van \textcite{priegoquesada2024} 
adviseren een horizontale afstand van circa 40–45\% van de lichaamslengte. Voor kinderen 
tussen 9--12 jaar met een gemiddelde lengte van \SIrange{135}{155}{\centi\meter} 
\parencite{dined2023} betekent dit een reikwijdte van ongeveer \SIrange{55}{70}{\centi\meter}.

\begin{table}[htbp]
\centering
\begin{threeparttable}
\caption{Antropometrische referentiewaarden voor kinderen 9--12 jaar \parencite{dined2023}.}
\label{tab:anthropometry}
\begin{tabularx}{\linewidth}{@{}l *{3}{S[table-format=3.0]}@{}}
\toprule
\textbf{Maat} & \textbf{Gemiddeld} & \textbf{Minimum} & \textbf{Maximum} \\
\midrule
Lichaamslengte    & 145 & 135 & 155 \\
Binnenbeenlengte  &  65 &  55 &  75 \\
Armlengte         &  60 &  55 &  65 \\
Schouderhoogte    &  95 &  85 & 105 \\
\bottomrule
\end{tabularx}
\begin{tablenotes}[para,flushleft]
\footnotesize Opmerking. Alle waarden in \si{\centi\meter}.
\end{tablenotes}
\end{threeparttable}
\end{table}


\begin{figure}[htbp]
  \centering
  \includegraphics[width=.65\linewidth]{Main/Chapters/Deelvraag/assets/zadelhoogte.png}
  \caption{Relatie tussen binnenbeenlengte en aanbevolen zadelhoogte.}
  \label{fig:zadelhoogte}
\end{figure}

\subsection{Conclusie}
De benodigde afmetingen voor comfortabel gebruik van de loopfiets zijn:
\begin{itemize}
    \item Zadelhoogte: \SIrange{58}{72}{\centi\meter}, afhankelijk van de binnenbeenlengte \parencite{savelberg2003,dined2023}.
    \item Stuurhoogte: \SIrange{80}{95}{\centi\meter}, voor een arm–romphoek van \SIrange{80}{100}{\degree} \parencite{conceicao2022}.
    \item Frame-lengte (zadel tot stuur): \SIrange{55}{70}{\centi\meter}, afhankelijk van lichaamslengte \parencite{priegoquesada2024,dined2023}.
\end{itemize}

Deze afmetingen vormen de ergonomische basis voor de dimensionering van de houten loopfiets.
