\section{Resultaten}
In Excel waren alle berekeningen gedaan o.a. de gemiddelden en de standard deviatie. Met de uitkomsten van deze berekeningen hebben we 2 grafieken gemaakt. Wij hebben een grafiek voor positie 1 (net van de grond af) en de andere grafiek voor positie 2 (bij de knie), zie \autoref{fig:L5_momenten}.
Bovendien hebben we de drie categorieën van tillen verdeeld in 3 soorten:
\begin{itemize}
    \item SquatBN (tussen de benen)
    \item SquatFN (voor de benen)
    \item Stoop (het voorover buigen)
\end{itemize}
Het nummer achter deze categorieën correspondeert met de hierboven beschreven meetposities. Categorieën met het nummer 1 verwijzen naar positie 1 (net van de grond af), weergegeven in \autoref{fig:L5POS1}. Categorieën met het nummer 2 verwijzen naar positie 2 (ter hoogte van de knie), weergegeven in \autoref{fig:L5POS2}.


\begin{figure}[H]
    \centering
    \begin{subfigure}{0.48\linewidth}
        \centering
        \includegraphics[width=\linewidth]{Main/Chapters/Deelvraag/assets/L5POS1.png}
        \caption{Moment rond L5 in positie 1}
        \label{fig:L5POS1}
    \end{subfigure}
    \hfill
    \begin{subfigure}{0.48\linewidth}
        \centering
        \includegraphics[width=\linewidth]{Main/Chapters/Deelvraag/assets/L5POS2.png}
        \caption{Moment rond L5 in positie 2}
        \label{fig:L5POS2}
    \end{subfigure}
    \caption{Gemiddeld moment rond L5 voor verschillende tiltechnieken in positie 1 en 2}
    \label{fig:L5_momenten}
\end{figure}
