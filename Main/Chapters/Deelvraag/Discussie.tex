\section{Discussie}

Vanuit ons onderzoek kunnen wij deduceren dat het squat tillen tussen je benen de meest voordelige manier van tillen is voor de lage rug. Het blijkt dat bij deze manier van tillen is het moment bij de lage rug het kleinst. Dit is ook te zien in de grafieken (\autoref{fig:L5POS1} en \autoref{fig:L5POS2}). De grootte van het moment is afhankelijk van de kracht vermenigvuldigd met de arm. Bij het squat tillen tussen je benen is het arm het kleinst, waardoor het moment ook kleiner is. Bij deze opdracht worden ook twee vragen gesteld:
\begin{enumerate}
    \item Bij welke techniek is het moment rond L5 het grootst in de startpositie?
    \item Bij welke techniek is het moment rond L5 het grootst in de tweede positie?
\end{enumerate}

Door onze berekening kunnen wij constateren dat het moment bij het stoop techniek het grootst is bij de startpositie. Maar bij het tweede positie is het moment het grootst bij de squat lift voor de knie. Zie onderstaande tabel (\autoref{tab:L5_momenten}).

\begin{table}[H]
\centering
\caption{Gemiddeld moment rond L5 per tiltechniek en positie}
\label{tab:L5_momenten}
\begin{tabular}{l c}
\toprule
\textbf{Categorie} & \textbf{Gemiddeld moment rond L5 (Nm)} \\
\midrule
SquatBN1 & 151.13 \\
SquatBN2 & 127.94 \\
SquatFN1 & 172.92 \\
SquatFN2 & 141.24 \\
Stoop1  & 179.06 \\
Stoop2  & 135.16 \\
\bottomrule
\end{tabular}
\end{table}

\subsection{Grafiek positie 1:}  
In dese grafiek kan je zien dat het tillen vanuit de stoop methode meer belastend is, maar wat meer opvalt is dat in deze grafieken ook is een verschil te zien tussen het moment tijdens de squat voor je knieën en tussen je knieën, maar waardoor komt dit verschil? 
Als je tussen je knieën tilt houd je het gewicht dichter bij je lichaam, de arm is dus korter. Als je een kleiner waarde voor de arm hebt in de formule: $M = F \cdot r$, waarbij ‘M‘ het moment is, ‘F’ de kracht en ‘r’ de arm. Komt er dus ook een kleinere waarde uit voor het moment. Daarom is het zo dat als je de squat doet met het gewicht voor je knieën er een grotere arm is en er dus meer moment nodig is (\autoref{fig:L5POS1}).

\subsection{Grafiek positie 2:}  
In positie 2 is er wat geks aan de hand. De stoop methode is hier minder belastend dan de squat met het gewicht voor de knieën. Hoe komt dit?
We hadden het net al over de arm, als deze groter is komt er ook een grotere waarde uit bij het moment. Positie 2 is wanneer het gewicht bij de knieën is. Bij de squat voor met het gewicht voor knieën ontstaat er iets wat niet bij de stoop methode gebeurt, namelijk een knie barrière. Je moet bij deze beweging nog het gewicht een stuk van je lichaam af bewegen om zo het kratje verder omhoog te krijgen, je krijgt anders een botsing met je knieën. Doordat je dit moet doen ontstaat er een grotere arm, terwijl je bij de stoop methode het kratje nog dichtbij je lichaam kan houden (\autoref{fig:L5POS2}).
