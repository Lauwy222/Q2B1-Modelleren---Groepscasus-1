\section{Werkwijze}
Voor het uitvoeren van dit onderzoek, hebben wij 4 proefpersonen gemeten. Elke proefpersoon moest een Velcro-pak aan dragen en daarop reflecterende markers erop plakken.  Deze zijn zodanig geplakt dat ze allemaal zichtbaar zijn vanuit het sagittale vlak. De punten waar markers werden geplakt is te vinden bij de onderstaande tabel (\autoref{tab:Lichaamsreferentiepunten}).
\begin{table}[H]
\centering
\begin{tabular}{|c|l|}
\hline
\textbf{Nr.} & \textbf{Omschrijving} \\ \hline
1  & end of hand \\ \hline
2  & wrist \\ \hline
3  & elbow \\ \hline
4  & shoulder \\ \hline
5  & toe \\ \hline
6  & heel \\ \hline
7  & ankle \\ \hline
8  & knee \\ \hline
9  & hip \\ \hline
10 & base of neck \\ \hline
11 & base of skull \\ \hline
12 & top of head \\ \hline
13 & cm load \\ \hline
\end{tabular}
\caption{Lichaamsreferentiepunten}
\label{tab:Lichaamsreferentiepunten}
\end{table}

De afstand tussen de knie en de enkel markers waren met een meetlint gemeten bij elke proefpersoon. Deze data werden later gebruikt om in Kinovea te kalibreren (\autoref{fig:Kalibratie}). 
\begin{figure}[H]
    \centering
    \includegraphics[width=0.25\linewidth]{Main//Chapters//Deelvraag//assets/Kalibratie.png}
    \caption{Kalibratie}
    \label{fig:Kalibratie}
\end{figure}
De proefpersonen moesten dan 3 verschillende til technieken uitvoeren voor het tillen van een kratje.  De 3 til technieken zijn: Stooplift (met de rug tillen), Squatlift (met de benen tillen). Het squat tillen werd uitgevoerd op 2 manieren. Eerst squat tillen met het kratje voor de benen. Als tweede het squat tillen tussen de benen. Het uitvoeren van deze til bewegingen werden gefilmd dan geïmporteerd en analyseert in Kinovea. 
We hebben erop gelet dat elke proefpersoon de til bewegingen zo identiek mogelijk uitvoerden zodanig dat we goede en vergelijkbare resultaten kregen.
\begin{enumerate}
    \item De stoop lift: je tilt alsof je in een stoel gaat zitten. Je buigt je knieën en heupen alsof je in een stoel gaat zitten, maar houd je rug recht.
    \item De squat voor knie: Je zorgt ervoor dat je tenen zo goed als tegen het kratje staat en maakt dan een squat beweging, je zorgt ervoor dat je een rechte rug houdt.
    \item De squat tussen knie: Je zorgt ervoor dat je voeten iets breder staan dan het kratje. Je maakt weer een squat beweging en zorgt voor een rechte rug.
\end{enumerate}
 
Ook hebben we tijdens het filmen erop gelet dat de proefpersoon de beweging rustig uitvoert, zodat we tijdens het analyseren een goed punt/frame konden uitkiezen om te analyseren.
Dit geeft wel een statisch beeld weer en dit zou ervoor zorgen dat er een foute waarde van de belasting uit zou komen, maar als je net zoals wij de beweging rustig/langzaam uitvoert zullen deze verschillen minimaal zijn (\autoref{fig:lifttechnieken}).

\begin{figure}[H]
    \centering
    \begin{subfigure}{0.32\linewidth}
        \centering
        \includegraphics[width=\linewidth]{Main/Chapters/Deelvraag/assets/Stooplift.png}
        \caption{Stooplift}
        \label{fig:Stooplift}
    \end{subfigure}
    \hfill
    \begin{subfigure}{0.32\linewidth}
        \centering
        \includegraphics[width=\linewidth]{Main/Chapters/Deelvraag/assets/Squatlift.png}
        \caption{Squatlift voor}
        \label{fig:Squatlift_voor}
    \end{subfigure}
    \hfill
    \begin{subfigure}{0.32\linewidth}
        \centering
        \includegraphics[width=\linewidth]{Main/Chapters/Deelvraag/assets/Squatlift_tussen.png}
        \caption{Squatlift tussen}
        \label{fig:Squatlift_tussen}
    \end{subfigure}
    \caption{Vergelijking van verschillende lifttechnieken}
    \label{fig:lifttechnieken}
\end{figure}

In Kinovea hebben we de markers gedigitaliseerd (zie \autoref{fig:Markers_Kinovea} en geexporteerd in Excel. Deze data werden dan in BT Lift Analysis (zie \autoref{fig:LiftAnalyse_tool}) gebruikt, om de momenten bij de lage rug te berekenen. 


\begin{figure}[H]
    \centering
    \begin{subfigure}{0.21\linewidth}
        \centering
        \includegraphics[width=\linewidth]{Main/Chapters/Deelvraag/assets/markers_kinovea.png}
        \caption{Markers Kinovea}
        \label{fig:Markers_Kinovea}
    \end{subfigure}
    \hfill
    \begin{subfigure}{0.5\linewidth}
        \centering
        \includegraphics[width=\linewidth]{Main//Chapters//Deelvraag//assets/moment-BT-LIFT-ANALYSIS.png}
        \caption{BT Lift Analysis tool}
        \label{fig:LiftAnalyse_tool}
    \end{subfigure}
    \caption{Analyseomgeving voor markerplaatsing en momentberekening}
    \label{fig:Analyse_tools}
\end{figure}

\subsection{Hoe hebben wij deze bewegingen geanalyseerd?}
1.	Als eerste is de afstand tussen enkel en knie van de proefpersoon is gemeten en gebruikt als referentiemaat.
2.	Vervolgens zijn de representatieve videoframes zijn geïmporteerd in Kinovea.
3.	In Kinovea hebben we het beeld gekalibreerd met behulp van de gemeten enkel-knie afstand. 
4.	Anatomische markers (enkel, knie, heup, L5, ect.) zijn geplaatst en benoemd.  
5.	De X en Y-coördinaten zijn in Excel geëxporteerd naar BT Lift Analysis (Spyder6)
6.	Waardes van Kinovea en BT Lift analysis zijn overgenomen in de “modelleren excel document”
7.	Waardes zijn met Excel functies berekend en gebruikt om grafieken te plotten.
